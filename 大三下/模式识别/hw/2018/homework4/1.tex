\documentclass{ctexart}
\usepackage{graphicx}
\usepackage{amsmath}
\usepackage{amsthm}
\usepackage{amssymb}
\usepackage{fancyhdr}
\usepackage{ifthen}
\usepackage{syntonly}
\usepackage[colorlinks, CJKbookmarks=true, linkcolor=red]{hyperref}
\pagestyle{plain}
\usepackage[raggedright]{titlesec}
\newtheorem{性质}{性质}
\newtheorem{定理}{定理}
\newtheorem{推论}{推论}
\begin{document}
已知$P(x=1|\omega_1)=p,P(x=1|\omega_2)=1-p,P(\omega_1)=P(\omega_2)=0.5$

所以$P(x=1)=P(x=1|\omega_1)P(\omega_1)+P(x=1|\omega_2)P(\omega_2)=0.5$,那么对于任何一个$d$维向量$x$,就有$P(x)=\frac{1}{2^d}$

我想算一下$P(\omega_1|x)$(不要问我算这个干嘛)

贝叶斯一下,$P(\omega_1|x)=\frac{P(x|\omega_1)P(\omega_1)}{P(x)}$

然后我用$\frac{1}{d}\sum\limits_{i=1}^d x_i$来估计$p$,就有$dp$个$1$,$d(1-p)$个$0$。

$P(x=1|\omega_1)=p$,自然有$P(x=0|\omega_1)=1-p$。

所以$P(x|\omega_1)=p^{dp}(1-p)^{d(1-p)}$

而$P(\omega_1)=\frac{1}{2},P(x)=\frac{1}{2^d}$。

所以就算出$P(\omega_1|x)=\frac{1}{2}[2p^p(1-p)^{1-p}]^d$。

注意到$d$趋于无穷,所以只要底数大于$1$概率就炸掉了。

我随便令$p=0.9$
\end{document}
